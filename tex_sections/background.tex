\section{Background}\label{sec:back}

\subsection{HYPER}
\subsection{Metagol}
Metagol is a system used for ILP which relies on meta-interpretative learning.\\
Using Metagol, four key components need to be defined:
\begin{itemize}
    \item \textbf{Metarules \((M)\).} Metarules are used to define the \emph{language bias} of the task. A large number metarules allows for less strict language bias, hence a larger search space in which to find a solution. 
    \item \textbf{Background Knowledge \((BK)\).} The knowledge the system is initially assumed to have about the task to be carried out. It is a set of Prolog rules that the system can use either directly or indirectly in order to induce the hypothesis.
    \item \textbf{Positive Examples \((E^+)\).}
    \item \textbf{Negative Examples \((E^-)\).}
\end{itemize}
With these four components defined, Metagol will try to find a solution running the following algorithm:
\begin{enumerate}
    \item Select a positive example to be proven.
    \item Try to prove the example using the existing \(BK\) or previously induced clauses.
    \item (If step 2 did not work) Unify the example with the head of a metarule and repeat steps 1,2 and 3 for each atom in the body of the obtained rule.
    \item Once the hypothesis is proven to be complete (all the positive examples have been proven and covered), test its consistency. If any negative example is covered, backtrack to a choice made in step 3 which, supposedly, led to this situation.
\end{enumerate}
In this brief illustration of Metagol, the process of \emph{predicate invention} is not covered due to a lack of time to further study it. Our findings about this process mainly derive
from experimental experience and have no theoretical backup. Nonetheless, we will still point out the influence it had on our results.


\subsection{ILASP}
