\subsection{ILASP}

ILASP enables learning programs containing normal rules, choice rules and both hard and weak constraints, these are the rules that compose ASP econdings and here this tool will be used for the maze problem. Weak constraints won't be covered, the goal here is to try to learn some normal, choiche and hard constraints for an encoding of that problem.

\subsubsection{Learning normal rules - learning how to walk}

The first task is learning to walk on the maze, considering adjacent cells and the obstacles (walls and co.). This task have been split for complexity reasons, as a result first it will be learned how to move on near cells and then obstacles will be considered, in 2 different ILASP scripts. Here some normal rules will be learned, in the next sections other kind of rules will be covered (regarding the same problem).

\subsubsection{Learning to walk on adjacent cells}

this is the ilasp code written for the pourpose, with some bit of background knowledge, definition of search space with language bias and some examples with all different "cases".  Finding "meaninful" examples have been pretty difficult.

\newpage
\begin{lstlisting}[language=Prolog, caption=Grid definition]
%%%%%%%%%%%%%%%%%%%%%%%%learn how to move on near cells
row(1..5).
col(1..5).

cell(X,Y) :- row(X), col(Y).

succ(0,1).
succ(X, X+1) :- cell(X,_).

%%%%%%%%%%%%%%%%%%%%%%%%%%%%%SEARCH_SPACE + EXAMPLES
#pos(p1, {next((4,2), (4,1)), next((4,2), (4,3)), next((4,2), (3,2)), next((4,2), (5,2))}, {}).
#pos(p2, {next((2,3), (2,2)), next((2,3), (1,3)), next((2,3), (2,4)), next((2,3), (3,3))}, {}).

%no out of range or jump
#neg(a, {next((1,0), (1,1))}, {}).
#neg(b, {next((1,1), (0,1))}, {}).
#neg(c, {next((0,1), (1,1))}, {}).
#neg(d, {next((1,1), (1,0))}, {}).
#neg(e, {next((5,5), (6,5))}, {}).
#neg(f, {next((5,5), (5,6))}, {}).
#neg(g, {next((6,5), (5,5))}, {}).
#neg(h, {next((5,6), (5,5))}, {}).
%no diagonal move
#neg(i, {next((2,4), (1,3))}, {}).
#neg(l, {next((2,4), (1,5))}, {}).
#neg(m, {next((2,4), (3,5))}, {}).
#neg(n, {next((2,4), (3,3))}, {}).
%no move same cell
#neg(o, {next((2,4), (2,4))}, {}).

#modeb(2, cell(var(r), var(c)), (positive, anti_reflexive)).
#modeb(1, succ(var(c), var(c)), (positive, anti_reflexive)).
#modeb(1, succ(var(r), var(r)), (positive, anti_reflexive)).
#modeh(next((var(r), var(c)), (var(r), var(c)))).

#maxv(3).


\end{lstlisting}