\subsection{Metagol}
Metagol is a system used for ILP which relies on meta-interpretative learning.\\
To shortly explain Metagol's learning procedure we will refer to our project, more
specifically to the file \texttt{Metagol/learn\_to\_walk.pl}, where the agent learns to
move from one cell to an adjacent, available one.\\
Metagol starts off by trying to prove one of the examples available, in our case we will
suppose it would pick the example \texttt{move((2,1),(3,1))}. Since the given background
knowledge cannot prove this atom and since there are yet no induced clauses, Metagol will
try to unify the atom with the head of one of the given metarules, which, in this case, are:
\begin{lstlisting}[language=Prolog, caption=Metarules in \texttt{learning\_to\_walk.pl}]
metarule(ident, [P,Q], [P,A,B], [[Q,A,B]]).
metarule(postcon, [P,Q,R], [P,A,B], [[Q,A,B], [R,B]]).
metarule(i_postcon, [P,Q,R], [P,A,B], [[R,B], [Q,A,B]]).  
\end{lstlisting}
We will later elaborate on why we are using two different versions of the postcondition metarule.
Supposing Metalog chose \texttt{i\_postcon} this would result in a substitution of its head
with the atom: