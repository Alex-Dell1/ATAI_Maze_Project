\subsection{Metagol}
The scripts discussed in this section of the report can be found in the \texttt{Metagol} folder.\\


\subsubsection{\texttt{learning\_to\_walk.pl}}
Since learning the predicate \texttt{adjacent/2} was quite trivial, the first task to be learned
consists moving from one cell to another adjacent, legal
one. Differently from the other two implementations, here the \texttt{adjacent/2} predicate is learned
indirectly through \emph{predicate invention}.\\
In this case, the system is assumed to know what a legal cell is and, given a pair of coordinates, how
to increment/decrease the value of a single coordinate.
\begin{lstlisting}[label={lst:bk_ltw}, language=Prolog, caption=Background Knowledge to walk, belowcaptionskip=1cm]
body_pred(inc_x/2).
body_pred(dec_x/2).
body_pred(inc_y/2).
body_pred(dec_y/2).
body_pred(legal_position/1).   
\end{lstlisting}
The positive and negative examples used in this program are quite straightforward to illustrate. A positive example
is needed for each possible direction (up, down, left, right). Having the predicate \texttt{legal\_position/1} as part 
of the background knowledge further simplifies the task of defining the negative examples, as we only need four of them:
one illegal move for each direction. This is possible because \texttt{legal\_position/1} is false whatever kind of illegal
position it is considering (out of bounds or obstacle).\\


The spotlight of this task is on metarules. Metarules define the search space and, therefore, they also have
a huge impact on both performance and the way the solution is presented. In this case the metarules used are
the following:
\begin{lstlisting}[language=Prolog, caption=Metarules in \texttt{learning\_to\_walk.pl}, belowcaptionskip=1cm]
metarule(ident, [P,Q], [P,A,B], [[Q,A,B]]). % Identity
metarule(postcon, [P,Q,R], [P,A,B], [[Q,A,B], [R,B]]). % Postcondition
metarule(i_postcon, [P,Q,R], [P,A,B], [[R,B], [Q,A,B]]). % Inverted postcondition  
\end{lstlisting}
Focussing on \texttt{postcon} and \texttt{i\_postcon}, it is possible to notice how they are basically defining
the same clause shape, just with two inverted atoms in the body. The results, though, will show how much difference
using one metarule or another can make.

\begin{lstlisting}[label={lst:postres},language=Prolog, caption=Result of \texttt{postcon}, belowcaptionskip=1cm]
move(A,B):-inc_x(A,B),legal_position(B).
move(A,B):-inc_y(A,B),legal_position(B).
move(A,B):-dec_x(A,B),legal_position(B).
move(A,B):-dec_y(A,B),legal_position(B).
\end{lstlisting}

\begin{lstlisting}[language=Prolog, caption=Result of \texttt{i\_postcon}, belowcaptionskip=1cm]
move(A,B):-legal_position(B),move_1(A,B).
move_1(A,B):-inc_x(A,B).
move_1(A,B):-inc_y(A,B).
move_1(A,B):-dec_x(A,B).
move_1(A,B):-dec_y(A,B).
\end{lstlisting}
As noticeable in Listing~\ref{lst:postres}, the solution is more succinct, with only 4 clauses
used for the solution. Using \texttt{i\_postcon} however, allows to learn the predicate \texttt{move\_1}
which corresponds to the predicate \texttt{adjacent/2} previously mentioned.\\

Before diving into the timing analysis for these two cases, a third one deserves to be mentioned where \texttt{legal\_position/1}
is replaced by the two conjuncts that defined it, namely \texttt{is\_free/1} (which checks whether a position is not an obstacle) and
\texttt{in\_range/1} (which checks if a position is in bounds).
In order to work on this task, a new metarule is introduced: \texttt{metarule([[P,Q,R,S],[P,A,B],[[Q,A,B],[R,B],[S,B]]).])}. This metarule,
to which we will refer to as \emph{double postcondition} (\texttt{double\_postcon}) shapes the resulting clause with two different postconditions.\\
The result is quite intuitive:
\begin{lstlisting}[language=Prolog, caption=Result of \texttt{double\_postcon} result, belowcaptionskip=1cm]
move(A,B):-inc_x(A,B),in_range(B),is_free(B).
move(A,B):-inc_y(A,B),in_range(B),is_free(B).
move(A,B):-dec_x(A,B),in_range(B),in_range(B).
move(A,B):-dec_y(A,B),in_range(B),is_free(B).
\end{lstlisting}

Finally, Table~\ref{tab:walk_res} offers an overview of the timings of these three slightly different implementations. These results highlight the impact that
one more clause or just a slightly increased clause length can have on time performance. Concluding, Metagol is affected by a trade-off between expressiveness and performance.
{\rowcolors{2}{gray!50!}{}
\begin{center}
    \begin{table}[h]
    \centering
    \begin{tabular}{ |c|c|c| } 
        \hline
        \texttt{postcon} & \texttt{i\_postcon} & \texttt{double\_postcon} \\ \hline
        0.047 & 0.121 & 0.156 \\ 
        \hline
    \end{tabular}
    \caption{\label{tab:walk_res}\texttt{learning\_to\_walk.pl} performances (seconds).}
    \end{table}
\end{center}
}

\subsubsection{\texttt{learning\_to\_travel\_with\_memory.pl}}\label{sec:lttm}
This program's task is to learn the predicate \texttt{reach(A,B,L)}, where \texttt{A} and \texttt{B} are cells and
\texttt{L} is a list containing the path from \texttt{A} to \texttt{B}.\\
In this case, the background knowledge already includes the predicate \texttt{move/2} previously learned with \texttt{learning\_to\_walk.pl}.
The metarules used are the following:
\begin{lstlisting}[label={lst:metaltm}, language=Prolog, caption=Metarules in \texttt{learning\_to\_travel\_with\_memory.pl}, belowcaptionskip=1cm]
metarule(recursion, [P,Q], [P,A,B,[A|L1]], [[Q,A,C], [P,C,B,L1]]).
metarule(ident, [P,Q], [P,A,B], [[Q,A,B]]).
metarule(ident2, [P,Q], [P,A,B,[A,B]], [[Q,A,B]]).
\end{lstlisting}
Focussing on the \texttt{recursion} metarule in Listing~\ref{lst:metaltm}, its recursive component lays in the fact that it enforces to reuse in its body the predicate used
in the head (\texttt{P}). It is fair to mention that no elaborated technique was used in order to understand which metarule would fit best for the given task. This metarule
was simply chosen because it represented the rule shape we would have used to \emph{"manually"} define \texttt{reach/3}.\\

One interesting thing about this implementation is about the number of examples being used: one positive example and no negative ones. This is because of how small
the search space is. Having the background knowledge only including \texttt{move/2} means already getting rid of any hypothesis of the predicate \texttt{reach/3} that
would allow illegal moves. To phrase it in a simpler way, an \emph{agent} that only knows how to move legally will not be able to make sequences of moves including
illegal ones.\\
Plus, not only is the search space quite small, the \emph{language bias} is also very restrictive: consider Metagol trying to prove the example \texttt{reach((1,1), (3,1), [(1,1), (2,1), (3,1)])}.
The only valid metarule to which unify this example is \texttt{recursion}.\\
Follows the result obtained from the implementation:
\begin{lstlisting}[language=Prolog, caption=Result of \texttt{learning\_to\_travel\_with\_memory.pl}, belowcaptionskip=1cm]
reach(A,B,[A,B]):-move(A,B).
reach(A,B,[A|C]):-move(A,D),reach(D,B,C).
\end{lstlisting}


\subsubsection{\texttt{reach\_from\_scratch\_memory.pl}}
This program's task is analogous to the one of the program illustrated at Section~\ref{sec:lttm}. This time, though, the
background knowledge is the same as for \texttt{learning\_to\_walk.pl}, shown in Listing~\ref{lst:bk_ltw}.\\
As a result of this, the system will have to deal with a large search space, this means that, at last, the spotlights points
onto the examples.\\
Given that the system now does not initially know how to \texttt{move/2}, the positive examples will have to include a successful
\emph{walk} from a legal cell to another adjacent one for each of the four directions. Last, only an example of one successful path
between two distant cells is needed. It is important for this path to contain moves in all four directions.
The negative examples are the same as it was for the first implementation described in Section~\ref{sec:lttm}: one illegal move for each
direction. The examples can be visualized more clearly in Listing~\ref{lst:ex_rfsm}.
\begin{lstlisting}[label={lst:ex_rfsm},language=Prolog, caption=Examples in \texttt{reach\_from\_scratch\_memory.pl}, belowcaptionskip=1cm]
Pos = [
    reach((1,1), (2,1), [(1,1), (2,1)]),
    reach((4,4), (4,3), [(4,4), (4,3)]),
    reach((5,2), (5,3), [(5,2), (5,3)]),
    reach((5,5), (4,5), [(5,5), (4,5)]),
    reach((3,1), (4,5), [(3,1), (4,1), (4,2), (4,3), (5,3), (5,4), (5,5), (4,5)])
],
Neg = [
    reach((5,1), (6,1), [(5,1), (6,1)]),
    reach((1,1), (1,0), [(1,1), (1,0)]),
    reach((5,5), (5,6), [(5,5), (5,6)]),
    reach((1,1), (0,1), [(1,1), (0,1)])
]
\end{lstlisting}
Surprisingly enough, no examples of paths containing incoherent moves were needed.
But as previously mentioned, this should be justifiable by the restricting language bias.
Following, the metarules (Listing~\ref{lst:mr_rfsm}) and the result of the implementation (Listing~\ref{lst:res_rfsm}).
\begin{lstlisting}[label={lst:mr_rfsm},language=Prolog, caption=Metarules in \texttt{reach\_from\_scratch\_memory.pl}, belowcaptionskip=1cm]
metarule(ident, [P,Q], [P,A,B], [[Q,A,B]]).
metarule(ident2, [P,Q], [P,A,B,[A,B]], [[Q,A,B]]).
metarule(postcon, [P,Q,R], [P,A,B], [[R,B], [Q,A,B]]).
metarule(recursion, [P,Q,R], [P,A,B,[A|L1]], [[R,B], [Q,A,C], [R,C], [P,C,B,L1]]).
\end{lstlisting}
\begin{lstlisting}[label={lst:res_rfsm},language=Prolog, caption=Result of \texttt{reach\_from\_scratch\_memory.pl}, belowcaptionskip=1cm]
reach(A,B,[A,B]):-reach_1(A,B).
reach_1(A,B):-legal_position(B),reach_2(A,B).
reach_2(A,B):-inc_x(A,B).
reach_2(A,B):-dec_y(A,B).
reach_2(A,B):-inc_y(A,B).
reach_2(A,B):-dec_x(A,B).
reach(A,B,[A|C]):-legal_position(B),reach_2(A,D),legal_position(D),reach(D,B,C).
\end{lstlisting}
On a last note, the metarule \texttt{recursion} in this case needed to be added a post-condition (on cell \texttt{B}) and a \emph{middle-condition}
(on cell \texttt{C}). It was quite hard to figure out why, but removing one of these atoms would lead to an endless execution of the program at a
number of clauses equal to 2.

\subsubsection{Other tasks}
\begin{itemize}
    \item \texttt{learn\_to\_win.pl}. Analogously to Cropper's example \texttt{robot.pl}, this program is able to offer the solution of the given Maze problem.
    \item \texttt{tail\_lttm.pl} (not working). This program has the same goal as the one described in Section~\ref{sec:lttm}, this time though the objective is to learn \texttt{reach/3} with tail recursion. The reason why this work was deemed relevant is offered at Section~\ref{sec:perf}.
\end{itemize}
\newpage