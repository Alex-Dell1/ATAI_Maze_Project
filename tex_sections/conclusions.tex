\section{Conclusions}\label{sec:conc}
Although many have already been offered in Section~\ref{sec:exp},
we would still like to add some more conclusions regarding the approach
that the system required with respect to the tasks.\\
Many of the times, the three systems required completely different approaches
to solve the encountered problems. This was particularly evident for ILASP, since,
differently from HYPER and Metagol, it is based on ASP. This is further proven when
considering the learning of predicate \texttt{reach/2}, as the paradigm of Answer Set
Programming makes this task extremely trivial and almost useless in the context of
Maze solving.\\
On a bright side, these differences in approach allowed us to better understand our
systems and exploit their traits. As for Metagol it immediately highlighted how powerful metarules can be for
time performances. As for HYPER and ILASP, it was possible to exploit their more general approach,
without having to define any metarules, although their sensitivity to the offered examples requires
quite a lot of attention.

