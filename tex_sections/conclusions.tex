\section{Conclusions}\label{sec:conc}
Although already many conclusions are offered in Section~\ref{sec:exp},
there are some more aspects to mention:\\

The three systems require completely different approaches
to solve the same problems encountered. This is particularly evident for ILASP, since,
unlike HYPER and Metagol, it is based on ASP. This is confirmed when
considering the learning of the predicate \texttt{reach/2}, as the paradigm of "Answer Set
Programming" makes this task extremely trivial and almost useless in the context of
solving the maze problem.\\

Comparing the different approaches allows to better understand the
systems and their traits.
As for Metagol it is immediately highlighted, how powerful metarules can be with regard to 
timing performances, when used correctly. However, this is not a trivial task, when there is no
initial idea on how the final result should look like. As for HYPER and ILASP, it was possible to exploit
their more general approaches as they do not require any metarules. Nevertheless, a lot of attention is required
when defining the examples given the sensitivity these systems have with respect to both positive and negative ones.
Which system should be chosen to implement a defined problem depends on the specific task and the requirements.

